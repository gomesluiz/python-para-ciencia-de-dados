\documentclass[t, final, 12pt, xcolor={usenames,dvipsnames}, table]{beamer}

\usepackage{listings}
\usepackage{color}
\usepackage[normalem]{ulem}
\useunder{\uline}{\ul}{}


 
\definecolor{codegreen}{rgb}{0,0.6,0}
\definecolor{codegray}{rgb}{0.5,0.5,0.5}
\definecolor{codepurple}{rgb}{0.58,0,0.82}
\definecolor{backcolour}{rgb}{0.95,0.95,0.92}

\lstset{
    basicstyle=\scriptsize\ttfamily,
    keywordstyle=\color{blue},
    identifierstyle=\color{black},
    commentstyle=\color{orange},
	stringstyle=\color{codepurple},
    showstringspaces=false,
    numberstyle=\color{gray}\tiny,
    breakatwhitespace=false,         
    breaklines=true,                 
    captionpos=b,                    
    keepspaces=true,                 
    numbers=left,                    
    numbersep=5pt,                  
    showspaces=false,                
    showstringspaces=false,
    showtabs=false,                  
    tabsize=2,
    extendedchars=\true,
    inputencoding=utf8,
    frame=tb, 
    columns=fixed,
    backgroundcolor=\color{red!32!green!33!blue!5},
    language=Python
}

\usetheme[pageofpages=of,
          alternativetitlepage=true,
          titlepagelogo=logo-telekinesis,
          ]{Torino}
          
\usecolortheme{freewilly}


\author{Luiz Alberto}
\title{Python para Ciência de Dados}
\institute{Ciência da Computação}
\date{\today}

% The log drawn in the upper right corner.
\logo{\includegraphics[height=0.125\paperheight]{logo-telekinesis}}

\begin{document}
  
\begin{frame}[t, fragile]{Tipos}
  \begin{itemize}
    \item String
    \begin{itemize}
      \item "Data Science", "Programming", "Python"
    \end{itemize}
    \item Inteiros
    \begin{itemize}
      \item -11, 7, 500, 700, 0, -80
    \end{itemize}
    \item Ponto flutuante
    \begin{itemize}
      \item 1.5, 0.5679, 2.909, -3.4560
    \end{itemize}
  \end{itemize}
\end{frame}

 
  \begin{frame}[t, fragile, allowframebreaks]{Strings}
  \begin{itemize}
    \item Um {\bf string} é uma cadeia de caracteres envolvida por aspas simples ou duplas
    \begin{itemize}
      \item "Data Science", 'Programming', "Python"
    \end{itemize}
    \item Mudando para letras maiúsculas e minúsculas
    \begin{lstlisting}[language=python]
      nome = "ada Lovelace"
      print(nome.title())
      Ada Lovelace
      print(nome.upper())
      ADA LOVELACE
      print(nome.lower())
      ada lovelace
    \end{lstlisting}
    \pagebreak
    \item Combinando ou concatenando strings
    \begin{lstlisting}[language=python]
      primeiro_nome = "ada"
      segundo_nome = "lovelace"
      nome_completo = primero_nome + " " + segundo_nome
      mensagem = "Ola  , " + nome_completo.title() + "!"
      print(mensagem)
      Boa tarde, Ada Lovelace!
    \end{lstlisting}
    \item Acrescentando espaços com tabulações ou quebras de linhas
    \begin{lstlisting}[language=python]
      print("\tAda Lovelace")
        Ada Lovelace
      print("Ada\nLovelace")
        Ada
        Lovelace
    \end{lstlisting}
    \pagebreak
    \item Removendo espaços em branco
    \begin{lstlisting}[language=python]
      nome = " Ada Lovelace " 
      print("["+nome.rstrip()+"]")
      [ Ada Lovelace]
      print("["+nome.rstrip()+"]")
      [Ada Lovelace ]
      print("["+nome.strip()+"]")
      [Ada Lovelace]
    \end{lstlisting}
  \end{itemize}
\end{frame}

 
  \begin{frame}[t, fragile]{Inteiros}
  \begin{itemize}
    \item Python trata números de várias maneiras diferentes
    \begin{lstlisting}[language=python]
      >>> 2 + 3
      5
      >>> 2 * 3 
      6
      >>> 3 / 2
      1.5 
      >>> 3 ** 2 
      9
      >>> 10 ** 6
      1000000
      >>> 2 + 3 * 4 
      14
      >>> (2 + 3) * 4
      20
    \end{lstlisting}
  \end{itemize}
\end{frame}
%
\begin{frame}[t, fragile]{Pontos flutuantes}
  \begin{itemize}
    \item Python chama qualquer número com um ponto decimal de {\it número de ponto flutuante} (float)
    \begin{lstlisting}[language=python]
      >>> 0.1 + 0.1
      0.2
      >>> 0.2 + 0.2
      0.4
      >>> 2 * 0.1
      0.2 
      >>> 2 * 0.2
      0.4
    \end{lstlisting}
  \end{itemize}
\end{frame}
%
\begin{frame}[t, fragile]{Convertendo números para string}
  \begin{lstlisting}[language=python]
    >>> mensagem = "Dia " + str(11) + "de maio."
    >>> print(mensagem)
    Dia 11 de maio.
  \end{lstlisting}
\end{frame}

 

 
  \begin{frame}[t, fragile]{Comentários}
  \begin{itemize}
    \item Um comentário permite escrever notas em seus programas em linguagem natural.
    \begin{lstlisting}[language=python]
      # o que a pessoas acham sobre o Python.
      print('Python is cool!')
    \end{lstlisting}
  \end{itemize}
\end{frame}
%

  \include{zen-do-python}
\end{document}

