\begin{frame}[t, fragile]{Inteiros}
  \begin{itemize}
    \item Python trata números de várias maneiras diferentes
    \begin{lstlisting}[language=python]
      >>> 2 + 3
      5
      >>> 2 * 3 
      6
      >>> 3 / 2
      1.5 
      >>> 3 ** 2 
      9
      >>> 10 ** 6
      1000000
      >>> 2 + 3 * 4 
      14
      >>> (2 + 3) * 4
      20
    \end{lstlisting}
  \end{itemize}
\end{frame}
%
\begin{frame}[t, fragile]{Pontos flutuantes}
  \begin{itemize}
    \item Python chama qualquer número com um ponto decimal de {\it número de ponto flutuante} (float)
    \begin{lstlisting}[language=python]
      >>> 0.1 + 0.1
      0.2
      >>> 0.2 + 0.2
      0.4
      >>> 2 * 0.1
      0.2 
      >>> 2 * 0.2
      0.4
    \end{lstlisting}
  \end{itemize}
\end{frame}
%
\begin{frame}[t, fragile]{Convertendo números para string}
  \begin{lstlisting}[language=python]
    >>> mensagem = "Dia " + str(11) + "de maio."
    >>> print(mensagem)
    Dia 11 de maio.
  \end{lstlisting}
\end{frame}

 

 