\begin{frame}[t, fragile, allowframebreaks]{Hora de colocar a mão na massa}
  \begin{enumerate}
    \item Crie uma lista, areas, que contém a area em metros quadrados do patio (11.25), da cozinha (18.0), da sala (20.0), do quarto (10.75) e do banheiro (9.50), nesta ordem. Utilize variáveis para cada area na criação da lista. Ao final exiba as areas.
    \item Altere a lista para colocar cada cômodo antes de sua area. Ao final exiba a nova lista.
    \item Exiba a segunda area, a área da sala e a última área.
    \item Exiba a soma das areas da cozinha e do quarto.
    \item Utilize o fatiamento para criar e exibir a lista, \verb!andares_de_baixo!, que contém os 6 primeiros elementos.
    \item Utilize o fatiamento para criar e exibir a lista, \verb!andares_de_baixo!, que contém os 4 últimos elementos. 
    \item Utilize o fatiamento para criar e exibir novamente a lista, \verb!andares_de_baixo!, que contém os 6 primeiros elementos. Omita o índice inicial.
    \item Utilize o fatiamento para criar e exibir novamente a lista, \verb!andares_de_baixo!, que contém os 4 últimos elementos. Omita o último índice.
    \item Mude a área do banheiro para 10.50 ao invés de 9.50
    \item Mude o nome da "sala" para "sala de jantar".
    \item Adicione a piscina cuja área é 24.5 no inicio da lista, e garagem cuja área 15.45 no final da lista.
    \item Remova da lista as informações do quarto (nome e area). Exiba a lista alterada
    \item Crie duas listas: uma com apenas os nomes dos campos e outra apenas com as áreas. Após isto ordene-as e exiba cada uma.
  \end{enumerate}
\end{frame}