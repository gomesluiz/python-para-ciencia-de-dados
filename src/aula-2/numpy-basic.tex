\begin{frame}[t, fragile]{NumPy}
  \begin{block}{\alert{Num}eric \alert{Py}thon:}
    Biblioteca para computação científica. Implementa arrays muldimensionais e permite a fácil execução de operações matemáticas  (\url{www.numpy.org}).
  \end{block}
  \begin{itemize}
    \item Utilizada em cálculos de alta performance de vetores e matrizes 
    \item Fornece versões pré-compiladas de funções numéricas
    \item Alternativa à Lista em Python: NumPy Array
    \item Cálculos sobre matrizes inteiras (broacasting)
    \item Escrita em C e Fortran
  \end{itemize}
\end{frame}
%
\begin{frame}[t, fragile]{NumPy}
  \lstinputlisting[language=python]{aula-2/codigos/numpy/numpy-basic-1.py}  
\end{frame}
%
\begin{frame}[t, fragile]{Comparação com listas}
  \lstinputlisting[language=python]{aula-2/codigos/numpy/numpy-basic-2.py}  
\end{frame}
%
\begin{frame}[t, fragile]{Subsetting}
  \lstinputlisting[language=python]{aula-2/codigos/numpy/numpy-basic-3.py}  
\end{frame}
%
\begin{frame}[t, fragile]{Arrays multidimensionais}
  \lstinputlisting[language=python]{aula-2/codigos/numpy/numpy-basic-4.py}  
\end{frame}
%
\begin{frame}[t, fragile, allowframebreaks]{Fatiamento}
  \lstinputlisting[language=python]{aula-2/codigos/numpy/numpy-basic-6.py}  
\end{frame}
%
\begin{frame}[t, fragile]{Estatística básica}
  \lstinputlisting[language=python]{aula-2/codigos/numpy/numpy-basic-7.py}  
\end{frame}
%
\begin{frame}[t, fragile]{Geração de dados}
  \lstinputlisting[language=python]{aula-2/codigos/numpy/numpy-basic-8.py}  
\end{frame}
%

 
 