\begin{frame}[t, fragile, allowframebreaks]{Hora de colocar a mão na massa}
  \begin{enumerate}
    \item Escreva uma cadeia \verb!if-elif-else! que determine o estágio da vida de uma pessoa. Leia o valor da variável idade e então:
    \begin{itemize}
      \item Se a pessoa tiver menos de 2 anos de idade, mostre a mensagem dizendo que ela é um bebê.
      \item Se a pessoa tiver pelo menos de 2 anos de idade, mas menos de 4, mostre a mensagem dizendo que ela é uma criança.
      \item Se a pessoa tiver pelo menos de 4 anos de idade, mas menos de 13, mostre a mensagem dizendo que ela é um(a) garot(a).
      \item Se a pessoa tiver pelo menos de 13 anos de idade, mas menos de 20, mostre a mensagem dizendo que ela é um(a) adolescente.
      \item Se a pessoa tiver pelo menos de 20 anos de idade, mas menos de 65, mostre a mensagem dizendo que ela é adulto.
      \item Se a pessoa tiver 65 anos de idade ou mais, mostre a mensagem dizendo que ela é idoso.
    \end{itemize}
    \item Faça o seguinte para criar um programa que simule o modo como os sites garantem que todos tenham um nome de usuário único.
    \begin{itemize}
      \item Crie uma lista chamada {\bf $usuarios\_correntes$} com cinco ou mais nomes dos usuários
      \item Crie outra lista chamada {\bf $usuarios\_novos$} com cinco nomes de usuários. Garanta que um ou dois dos novos usuários também estejam na lista {\bf $usuarios\_correntes$}.
      \item Percorra a lista {\bf $usuarios\_novos$} com um laço para ver se cada novo nome de usuário já foi usado. Em caso afirmativo, coloque esse nome na lista $usuarios_repetidos$ e mostre a mensagem informando que o nome não está disponível. Se um nome de usuário não foi usado, apresente uma mensagem dizendo que o nome do usuário está disponível.
      \item Certifique-se de que sua comporação não levará em conta as diferenças entre maiúsculas e minúsculas.
    \end{itemize}
  \end{enumerate}
\end{frame}