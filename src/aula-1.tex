\documentclass[t, final, 12pt, xcolor={usenames,dvipsnames}, table]{beamer}

\usepackage{listings}
\usepackage{color}
\usepackage[normalem]{ulem}
\useunder{\uline}{\ul}{}


 
\definecolor{codegreen}{rgb}{0,0.6,0}
\definecolor{codegray}{rgb}{0.5,0.5,0.5}
\definecolor{codepurple}{rgb}{0.58,0,0.82}
\definecolor{backcolour}{rgb}{0.95,0.95,0.92}

\lstset{
    basicstyle=\scriptsize\ttfamily,
    keywordstyle=\color{blue},
    identifierstyle=\color{black},
    commentstyle=\color{orange},
	stringstyle=\color{codepurple},
    showstringspaces=false,
    numberstyle=\color{gray}\tiny,
    breakatwhitespace=false,         
    breaklines=true,                 
    captionpos=b,                    
    keepspaces=true,                 
    numbers=left,                    
    numbersep=5pt,                  
    showspaces=false,                
    showstringspaces=false,
    showtabs=false,                  
    tabsize=2,
    extendedchars=\true,
    inputencoding=utf8,
    frame=tb, 
    columns=fixed,
    backgroundcolor=\color{red!32!green!33!blue!5},
    language=Python
}

\usetheme[pageofpages=of,
          alternativetitlepage=true,
          titlepagelogo=figuras/python-logo-big,
          ]{Torino}
          
\usecolortheme{freewilly}


\author{Luiz Alberto}
\title{Python para Ciência de Dados}
\subtitle{Fundamentos da Linguagem Python}
\institute{Ciência da Computação}
\date{\today}

% The log drawn in the upper right corner.
\logo{\includegraphics[height=0.125\paperheight]{figuras/python-logo-small}}

\begin{document}
  \begin{frame}[t,plain]
    \titlepage
  \end{frame}
  
  
  \begin{frame}[t, fragile]{Conteúdo do Curso}
  \begin{itemize}
    \item Fundamentos da linguagem
    \item Controle de decisão e repetição
    \item Listas e dicionários
    \item Funções e pacotes
    \item Manipulação de dados com o Pandas
    \item Processamento numérico com Numpy
    \item Visualização de dados com Matplotlib
    \item Análise exploratória de dados
  \end{itemize}
\end{frame}

 
  \begin{frame}[t, fragile]{Conteúdo do Curso}
  \begin{itemize}
    \item 8h -- 12h20 e das 13h10 --17h50
    \item Intervalo para lanche e almoço
  \end{itemize}
\end{frame}
  
\begin{frame}[t, fragile]{Ciência da Dados}
  \begin{itemize}
    \item Habilidade de manipular, analisar e extrair valor dos dados
    \item Combina métodos de diversas áreas como estatística, análise de dados, computação
    \item Tem o objetivo de encontrar padrões, tendências, fazer predições a fim de agregar valor para o negócio
  \end{itemize}
\end{frame}

 
  \begin{frame}[t, fragile]{Python}
  \begin{itemize}
    \item Linguagem open-source de uso geral
    \item Orientada a objetos, funcional e procedural
    \item Interface com C/C++/Java/Fortran
    \item Diversas bibliotecas para Ciência de Dados
  \end{itemize}
\end{frame}
%
\begin{frame}[t, fragile, allowframebreaks]{Bibliotecas para Ciência de Dados}
  \begin{itemize}
    \item {\bf NumPy}: biblioteca para computação científica. Implementa arrays multidimensionais e permite a fácil execução de operações matemáticas e lógicas
    \item {\bf Matplotlib}: biblioteca para visualização e plotagem de gráficos em duas dimensões, como histogramas, barras e pizza.
    \item {\bf Pandas}: biblioteca para análise de dados. Fornece ferramentas para a manipulação de estruturas de dados, como matrizes, vetores e dataframes
    \item {\bf Scikit-learn}: biblioteca para Aprendizado de Máquina (Machine Learning). Fornece diversos algoritmos implementados, métodos de análise e processamento de dados, métricas de avaliação
    \item {\bf NLTK}: biblioteca para processamento de textos e linguagem natural. Fornece um conjunto de bibliotecas de processamento de texto para classificação, tokenização, stemming e tagging 
  \end{itemize}
\end{frame}
%

 
  %\begin{frame}[t, fragile]{Ambiente de Desenvolvimento}
  
\end{frame}
 
  
\begin{frame}[t, fragile]{Tipos}
  \begin{itemize}
    \item String
    \begin{itemize}
      \item "Data Science", "Programming", "Python"
    \end{itemize}
    \item Inteiros
    \begin{itemize}
      \item -11, 7, 500, 700, 0, -80
    \end{itemize}
    \item Ponto flutuante
    \begin{itemize}
      \item 1.5, 0.5679, 2.909, -3.4560
    \end{itemize}
  \end{itemize}
\end{frame}

 
  \begin{frame}[t, fragile]{Entrada do Usuário}
  \lstinputlisting[language=python]{codigos/entrada/entrada1.py}
\end{frame}

 
  \begin{frame}[t, fragile, allowframebreaks]{Strings}
  \begin{itemize}
    \item Um {\bf string} é uma cadeia de caracteres envolvida por aspas simples ou duplas
    \begin{itemize}
      \item "Data Science", 'Programming', "Python"
    \end{itemize}
    \item Mudando para letras maiúsculas e minúsculas
    \begin{lstlisting}[language=python]
      nome = "ada Lovelace"
      print(nome.title())
      Ada Lovelace
      print(nome.upper())
      ADA LOVELACE
      print(nome.lower())
      ada lovelace
    \end{lstlisting}
    \pagebreak
    \item Combinando ou concatenando strings
    \begin{lstlisting}[language=python]
      primeiro_nome = "ada"
      segundo_nome = "lovelace"
      nome_completo = primero_nome + " " + segundo_nome
      mensagem = "Ola  , " + nome_completo.title() + "!"
      print(mensagem)
      Boa tarde, Ada Lovelace!
    \end{lstlisting}
    \item Acrescentando espaços com tabulações ou quebras de linhas
    \begin{lstlisting}[language=python]
      print("\tAda Lovelace")
        Ada Lovelace
      print("Ada\nLovelace")
        Ada
        Lovelace
    \end{lstlisting}
    \pagebreak
    \item Removendo espaços em branco
    \begin{lstlisting}[language=python]
      nome = " Ada Lovelace " 
      print("["+nome.rstrip()+"]")
      [ Ada Lovelace]
      print("["+nome.rstrip()+"]")
      [Ada Lovelace ]
      print("["+nome.strip()+"]")
      [Ada Lovelace]
    \end{lstlisting}
  \end{itemize}
\end{frame}

 
  \begin{frame}[t, fragile]{Inteiros}
  \begin{itemize}
    \item Python trata números de várias maneiras diferentes
    \begin{lstlisting}[language=python]
      >>> 2 + 3
      5
      >>> 2 * 3 
      6
      >>> 3 / 2
      1.5 
      >>> 3 ** 2 
      9
      >>> 10 ** 6
      1000000
      >>> 2 + 3 * 4 
      14
      >>> (2 + 3) * 4
      20
    \end{lstlisting}
  \end{itemize}
\end{frame}
%
\begin{frame}[t, fragile]{Pontos flutuantes}
  \begin{itemize}
    \item Python chama qualquer número com um ponto decimal de {\it número de ponto flutuante} (float)
    \begin{lstlisting}[language=python]
      >>> 0.1 + 0.1
      0.2
      >>> 0.2 + 0.2
      0.4
      >>> 2 * 0.1
      0.2 
      >>> 2 * 0.2
      0.4
    \end{lstlisting}
  \end{itemize}
\end{frame}
%
\begin{frame}[t, fragile]{Convertendo números para string}
  \begin{lstlisting}[language=python]
    >>> mensagem = "Dia " + str(11) + "de maio."
    >>> print(mensagem)
    Dia 11 de maio.
  \end{lstlisting}
\end{frame}

 

 
  \begin{frame}[t, fragile]{Comentários}
  \begin{itemize}
    \item Um comentário permite escrever notas em seus programas em linguagem natural.
    \begin{lstlisting}[language=python]
      # o que a pessoas acham sobre o Python.
      print('Python is cool!')
    \end{lstlisting}
  \end{itemize}
\end{frame}
%

  \include{zen-do-python}
  \begin{frame}[t, fragile, allowframebreaks]{Hora de colocar a mão na massa}
  Salvar cada um dos exercícios a seguir em um arquivo separado.
  \begin{itemize}
    \item Armazene o nome de uma pessoa lido pelo teclado, inclua alguns caracteres em branco no início e no final do nome. Exiba o nome uma vez, de modo que os espaços em branco em torno do nome sejam mostrados. Em seguida, exiba o nome usando cada uma das três funções de remoção de espaço: {\bf lstrip(), rstrip(), strip()}.
    \item No exercício anterior, exiba o nome com as letras em minúsculo, em maiúsculo e com as iniciais em maiúsculo.
    \item Escreva um programa que pergunte o nome e a idade do usuário. Exiba uma mensagem dizendo em qual ano ele terá 100 anos.
    \item Escreva um programa que receba o saldo de uma aplicação bancária, a taxa de correção e a quantidade de anos o usuário deixara o dinheiro aplicado. Ao final o programa deverá exibir o saldo final do usuário.
    \item Escreva um programa que receba a altura e o peso de uma pessoa e forneça o seu IMC.
  \end{itemize}
\end{frame}
  
\begin{frame}[t, fragile]{Introdução às Listas}
  \begin{itemize}
    \item Coleção de itens em uma ordem particular
    \item Itens podem ser letras, dígitos e etc.
    \begin{itemize}
      \item não precisam estar relacionados de nenhum modo em particular
    \end{itemize}
    \item Em Python, colchetes([]) indicam uma lista, e elementos individuais da
      lista são separados por vírgula
    \item A posição dos índices de uma lista começa em 0 e não em 1
  \end{itemize}
\end{frame}
%
\begin{frame}[t, fragile, allowframebreaks]{Acessando elementos da lista}
  \begin{itemize}
    \item Escreva o nome da lista seguido do índice do item entre colchetes
  
    \begin{lstlisting}[language=python]
      >>> mestres = ['yoda', 'qui-gon', 'kenobi'
          , 'luke']
      >>> print(mestres[0])
      yoda
      >>> print(mestres[0].title())
      Yoda
    \end{lstlisting}
    \item Python tem uma sintaxe especial para acessar o último elemento de uma lista
    \begin{lstlisting}[language=python]
      >>> print(mestres[-1].title())
      Luke
    \end{lstlisting}
  \end{itemize}
\end{frame}
%
\begin{frame}[t, fragile, allowframebreaks]{Acrescentando elementos no final da lista}
  \begin{itemize}
    \begin{lstlisting}[language=python]
      >>> mestres = ['yoda', 'qui-gon', 'kenobi'
          , 'luke']
      >>> mestres.append('windu')
      >>> print(mestres)
      ['yoda', 'qui-gon', 'kenobi', 'luke', 'windu']
    \end{lstlisting}
    \item O método {\bf append()} acrescenta o elemento 'windu' no final da lista 
    \pagebreak
    \item O método {\bf append()} facilita a criação de listas dinamicamente
    \begin{lstlisting}[language=python]
      casas = []
      casas.append('starks')
      casas.append('greyjoy')
      casas.append('tyrell')
      casas.append('lannister')
      print(casas)
      ['starks', 'greyjoy', 'tyrell', 'lannister']
    \end{lstlisting}
  \end{itemize}
\end{frame}
%
\begin{frame}[t, fragile]{Inserindo elementos em uma lista}
  \begin{itemize}
    \item O método {\bf insert} permite adicionar um elemento em qualquer posição da lista
    \begin{lstlisting}[language=python]
      mestres = ['yoda', 'qui-gon', 'kenobi', 'luke']
      mestres.insert(0, 'vader')
      print(mestres)
      ['vader', 'yoda', 'qui-gon', 'kenobi', 'luke']
    \end{lstlisting} 
  \end{itemize}
\end{frame}
%
\begin{frame}[t, fragile]{Removendo elementos da com {\bf del}}
  \begin{itemize}
    \item O comando {\bf del} remove item da lista 
    \begin{lstlisting}[language=python]
      mestres = ['yoda', 'qui-gon', 'kenobi', 'luke']
      del(mestres[0])
      print(mestres)
      ['qui-gon', 'kenobi', 'luke']
    \end{lstlisting} 
    \item Não é possível reutilizar o elemento removido
  \end{itemize}
\end{frame}
%
\begin{frame}[t, fragile]{Removendo elementos da lista com {\bf pop}}
  \begin{itemize}
    \item O método {\bf pop} remove o último item da lista por padrão, sendo que elemento removido pode ser reutilizado
    \lstinputlisting[language=python]{codigos/listas/listas6.py}
    \item De fato, o método {\bf pop} pode ser utilizado para remover qualquer elemento da lista, basta passar o índice do elemento
    \lstinputlisting[language=python]{codigos/listas/listas7.py}
  \end{itemize}
  
\end{frame}
%
%  \begin{frame}[t, fragile]{Removendo elementos da lista com {\bf remove}}
%    \begin{itemize}
%      \item O método {\bf remove} um elemento de acordo com seu valor.
%      \lstinputlisting[language=python]{codigos/listas/listas8.py}
%    \end{itemize}  
%  \end{frame}
%
\begin{frame}[t, fragile]{Ordenando a lista com {\bf sort}}
  \begin{itemize}
    \item O método {\bf sort} altera a forma da lista permanentemente 
    \lstinputlisting[language=python]{codigos/listas/listas9.py}
    \item O método {\bf sort} permite ordenar a lista em ordem alfabética inversa
    \lstinputlisting[language=python]{codigos/listas/listas10.py}
  \end{itemize}  
\end{frame}
%
\begin{frame}[t, fragile]{Ordenando a lista com {\bf sorted}}
  \begin{itemize}
    \item O método {\bf sorted} mantém a forma original da lista
    \lstinputlisting[language=python]{codigos/listas/listas11.py}
  \end{itemize}  
\end{frame}
%
\begin{frame}[t, fragile]{Exibindo uma lista em ordem inversa com {\bf reverse}}
  \begin{itemize}
    \item O método {\bf reverse} mantém a forma original da lista
    \lstinputlisting[language=python]{codigos/listas/listas12.py}
  \end{itemize}  
\end{frame}
%






 
  \begin{frame}[t, fragile, allowframebreaks]{Hora de colocar a mão na massa}
  \begin{enumerate}
    \item Crie uma lista, areas, que contém a area em metros quadrados do patio (11.25), da cozinha (18.0), da sala (20.0), do quarto (10.75) e do banheiro (9.50), nesta ordem. Utilize variáveis para cada area na criação da lista. Ao final exiba as areas.
    \item Altere a lista para colocar cada cômodo antes de sua area. Ao final exiba a nova lista.
    \item Exiba a segunda area, a área da sala e a última área.
    \item Exiba a soma das areas da cozinha e do quarto.
    \item Utilize o fatiamento para criar e exibir a lista, \verb!andares_de_baixo!, que contém os 6 primeiros elementos.
    \item Utilize o fatiamento para criar e exibir a lista, \verb!andares_de_baixo!, que contém os 4 últimos elementos. 
    \item Utilize o fatiamento para criar e exibir novamente a lista, \verb!andares_de_baixo!, que contém os 6 primeiros elementos. Omita o índice inicial.
    \item Utilize o fatiamento para criar e exibir novamente a lista, \verb!andares_de_baixo!, que contém os 4 últimos elementos. Omita o último índice.
    \item Mude a área do banheiro para 10.50 ao invés de 9.50
    \item Mude o nome da "sala" para "sala de jantar".
    \item Adicione a piscina cuja área é 24.5 no inicio da lista, e garagem cuja área 15.45 no final da lista.
    \item Remova da lista as informações do quarto (nome e area). Exiba a lista alterada
    \item Crie duas listas: uma com apenas os nomes dos campos e outra apenas com as áreas. Após isto ordene-as e exiba cada uma.
  \end{enumerate}
\end{frame}
  
\begin{frame}[t, fragile]{Percorrendo uma lista com um laço}
  \begin{itemize}
    \item Um laço {\bf for} permite que se percorra uma lista de início até o final
    \lstinputlisting[language=python]{codigos/listas/listas13.py}
    \item a linha 2 deverá estar {\bf indentada} para ser executada pelo laço for
  \end{itemize}
\end{frame}
%
\begin{frame}[t, fragile, allowframebreaks]{Criando listas numéricas}
  \begin{itemize}
    \item A função {\bf range} permite a geração de uma série de números
    \lstinputlisting[language=python]{codigos/listas/listas14.py}
     \item A  função range {\bf NÃO} exibe o limite superior do intervalo.
     \item Um salto pode ser fornecido para a função {\bf range} para que a função ignore alguns números no intervalo
    \lstinputlisting[language=python]{codigos/listas/listas15.py}
\pagebreak
    \item Outro Exemplo: Imprimindo os 10 primeiros quadrados perfeitos
    \lstinputlisting[language=python]{codigos/listas/listas16.py}
  \end{itemize}
\end{frame}
%
\begin{frame}[t, fragile, allowframebreaks]{Estatísticas simples com lista de números}
  \begin{itemize}
    \item {\bf min()}, {\bf max()} e {\bf sum()} são funções específicas para listas de números:
    \lstinputlisting[language=python]{codigos/listas/listas17.py}
  \end{itemize}
\end{frame}
%
\begin{frame}[t, fragile]{List comprehensions}
  \begin{itemize}
    \item Combina o laço {\bf for} e a criação de novos elementos em uma lista, e concatena cada novo elemento automaticamente
    \lstinputlisting[language=python]{codigos/listas/listas18.py}
  \end{itemize}
\end{frame}
%
\begin{frame}[t, fragile, allowframebreaks]{Fatiando uma lista}
  \begin{itemize}
    \item É necessário especificar o índice do primeiro e do último elemento desejado
    \lstinputlisting[language=python]{codigos/listas/listas19.py}
    \item O intervalo 0:3 faz com os elementos 0, 1 e 2 sejam impressos
    \item Se o primeiro índice de uma fatia for omitido, Python comecará automaticamente do inicio da lista
    \lstinputlisting[language=python]{codigos/listas/listas20.py}
    \item Todos os elementos a partir de qualquer posição podem ser apresentados, até mesmo a partir do final
    \lstinputlisting[language=python]{codigos/listas/listas21.py}
  \end{itemize}
\end{frame}
%
\begin{frame}[t, fragile]{Percorrendo uma fatia com um laço}
  \begin{itemize}
    \item Pode-se utiliza um laço {\bf for} para percorrer os elementos de uma fatia:                                                                                                                                                  
    \lstinputlisting[language=python]{codigos/listas/listas22.py}
  \end{itemize}
\end{frame}
%
\begin{frame}[t, fragile]{Copiando uma lista}
  \begin{itemize}
    \item Pode-se criar uma fatia que inclua a lista original inteira omitindo o primeiro e o segundo índice ([:])
    \lstinputlisting[language=python]{codigos/listas/listas23.py}
    \item \alert{Isto não funciona!}
    \lstinputlisting[language=python]{codigos/listas/listas24.py}
  \end{itemize}  
\end{frame}
%





 
  \begin{frame}[t, fragile, allowframebreaks]{Hora de colocar a mão na massa}
  Considerando a lista de areas do cômodos criada anteriormente, resolva os exercícios abaixo:
  \begin{enumerate}
    \item Exiba todos os elementos da lista areas.
    \item Exiba de linha em linha cada par cômodo e sua area.
    \item Exiba a maior area, a menor area e a média das áreas.
    \item Utilize o fatiamento para criar e exibir a lista, \verb!andares_de_baixo!, que contém os 6 primeiros elementos.
    \item Utilize o fatiamento para criar e exibir a lista, \verb!andares_de_baixo!, que contém os 4 últimos elementos. 
    \item Utilize o fatiamento para criar e exibir novamente a lista, \verb!andares_de_baixo!, que contém os 6 primeiros elementos. Omita o índice inicial.
    \item Utilize o fatiamento para criar e exibir novamente a lista, \verb!andares_de_baixo!, que contém os 4 últimos elementos. Omita o último índice.
    \item A partir da lista de anos de nascimento, crie uma lista com as idades utilizando {\bf list comprehensions}
  \end{enumerate}
\end{frame}
  \begin{frame}[t, fragile, allowframebreaks]{Tuplas}
  \begin{itemize}
    \item Tuplas são listas {\it imutáveis } em Python. Exemplo:
      \lstinputlisting[language=python]{codigos/listas/listas25.py}
    \item Erro ao tentar alterar o conteúdo da uma tupla
      \lstinputlisting[language=python]{codigos/listas/listas26.py}
    \item Percorrendo todos os valores de uma tupla com um laço
      \lstinputlisting[language=python]{codigos/listas/listas27.py}
  \end{itemize}  
\end{frame}
%




 
  \begin{frame}[t, fragile, allowframebreaks]{Hora de colocar a mão na massa}
  \begin{enumerate}
    \item Um restaurante do tipo buffet oferece apenas cinco tipos básicos de comida. Pense em cinco pratos simples e armazene-os em uma tupla.
    \item Use um laço {\bf for} para exibir cada prato oferecido pelo restaurante.
    \item Tente modificar um dos itens e certifique-se de que Python rejeita a mudança.
    \item O restaurante muda o seu cardápio, substituindo dois dos itens com pratos diferentes. Acrescente um bloco de código que reescreva a tupla e, em seguida, use um laço for para exibir cada um dos itens do cardápio revisado.
  \end{enumerate}
\end{frame}
  \begin{frame}[t, fragile, allowframebreaks]{Instrução condicional IF}
  \begin{itemize}
    \item Testes condicionais
      \lstinputlisting[language=python]{codigos/listas/listas29.py}
    \item Exemplo 1:
      \lstinputlisting[language=python]{codigos/listas/listas28.py}
    \item Exemplo 2:
      \lstinputlisting[language=python]{codigos/listas/listas30.py}
  \end{itemize}  
\end{frame}
%
\begin{frame}[t, fragile, allowframebreaks]{Instrução IF com listas}
  \begin{itemize}
    \item Verificando se uma lista está vazia
      \lstinputlisting[language=python]{codigos/listas/listas31.py}
      \pagebreak
    \item Utilizando diversas listas:
      \lstinputlisting[language=python]{codigos/listas/listas32.py}
  \end{itemize}  
\end{frame}
%




 
  \begin{frame}[t, fragile, allowframebreaks]{Hora de colocar a mão na massa}
  \begin{enumerate}
    \item Escreva uma cadeia \verb!if-elif-else! que determine o estágio da vida de uma pessoa. Leia o valor da variável idade e então:
    \begin{itemize}
      \item Se a pessoa tiver menos de 2 anos de idade, mostre a mensagem dizendo que ela é um bebê.
      \item Se a pessoa tiver pelo menos de 2 anos de idade, mas menos de 4, mostre a mensagem dizendo que ela é uma criança.
      \item Se a pessoa tiver pelo menos de 4 anos de idade, mas menos de 13, mostre a mensagem dizendo que ela é um(a) garot(a).
      \item Se a pessoa tiver pelo menos de 13 anos de idade, mas menos de 20, mostre a mensagem dizendo que ela é um(a) adolescente.
      \item Se a pessoa tiver pelo menos de 20 anos de idade, mas menos de 65, mostre a mensagem dizendo que ela é adulto.
      \item Se a pessoa tiver 65 anos de idade ou mais, mostre a mensagem dizendo que ela é idoso.
    \end{itemize}
    \item Faça o seguinte para criar um programa que simule o modo como os sites garantem que todos tenham um nome de usuário único.
    \begin{itemize}
      \item Crie uma lista chamada {\bf $usuarios\_correntes$} com cinco ou mais nomes dos usuários
      \item Crie outra lista chamada {\bf $usuarios\_novos$} com cinco nomes de usuários. Garanta que um ou dois dos novos usuários também estejam na lista {\bf $usuarios\_correntes$}.
      \item Percorra a lista {\bf $usuarios\_novos$} com um laço para ver se cada novo nome de usuário já foi usado. Em caso afirmativo, coloque esse nome na lista $usuarios_repetidos$ e mostre a mensagem informando que o nome não está disponível. Se um nome de usuário não foi usado, apresente uma mensagem dizendo que o nome do usuário está disponível.
      \item Certifique-se de que sua comporação não levará em conta as diferenças entre maiúsculas e minúsculas.
    \end{itemize}
  \end{enumerate}
\end{frame}  
  \begin{frame}[t, fragile]{Dicionários}
  \begin{itemize}
      \item Dicionários permitem conectar informações {\bf relacionadas} e modelar uma diversidade de objetos do mundo real
      \begin{itemize}
      \item podemos armazenar, por exemplo, o nome, ano, cor e fabricante de uma caro em uma única estrutura, ao invés de quatro estruturas separadas
     \end{itemize}
     \item Um dicionário é uma coleção de pares {\it chave-valor}. Cada chave é conectada a um valor:
     \lstinputlisting[language=python]{codigos/listas/listas33.py}
  \end{itemize}  
\end{frame}
%
\begin{frame}[t, fragile]{Acessando valores do dicionário}
  \lstinputlisting[language=python]{codigos/listas/listas34.py}
\end{frame}
%
\begin{frame}[t, fragile]{Començando um dicionário vazio}
  \lstinputlisting[language=python]{codigos/listas/listas35.py}
\end{frame}
%
\begin{frame}[t, fragile]{Um exemplo de utilização de dicionário}
  \lstinputlisting[language=python]{codigos/listas/listas36.py}
\end{frame}
%
\begin{frame}[t, fragile]{Percorrendo todos os pares chave-valor}
  \lstinputlisting[language=python]{codigos/listas/listas37.py}
\end{frame}
%
\begin{frame}[t, fragile]{Percorrendo todos as chaves}
  \lstinputlisting[language=python]{codigos/listas/listas38.py}
\end{frame}
%
\begin{frame}[t, fragile]{Percorrendo todos as chaves em ordem}
  \lstinputlisting[language=python]{codigos/listas/listas39.py}
\end{frame}
%
\begin{frame}[t, fragile]{Percorrendo todos os valores}
  \lstinputlisting[language=python]{codigos/listas/listas40.py}
\end{frame}
%
\begin{frame}[t, fragile]{Uma lista de dicionários}
  \lstinputlisting[language=python]{codigos/listas/listas41.py}
\end{frame}
%
\begin{frame}[t, fragile, allowframebreaks]{Uma lista em um dicionário}
  \lstinputlisting[language=python]{codigos/listas/listas42.py}
\end{frame}
%
\begin{frame}[t, fragile, allowframebreaks]{Um dicionário em um dicionário}
  \lstinputlisting[language=python]{codigos/listas/listas43.py}
\end{frame}
%


 
  \begin{frame}[t, fragile, allowframebreaks]{Hora de colocar a mão na massa}
  \begin{enumerate}
    \item Crie um dicionário com a frequência de palavras em uma determinada lista. Exiba a palavra com a maior frequência.
    \item Um bloco de ações negociadas publicamente tem uma variedade de atributos. Uma ação tem um símbolo e um nome da empresa. Crie um dicionário com os símbolos e nomes da empresas. Por exemplo: {\bf acoes = \{'GM': 'General Motors', 'CAT': 'Caterpillar', 'EK': 'Eastman Kodak'\}}. 
    \item Agora crie uma lista de compras de ações. Cada elemento da lista é tupla contendo simbolo, preco, data e quantidade de ações. Por exemplo: {\bf compras = [('GM', 100, '10/09/2001', 48), ('CAT', 100, '01/04/1999', 24), ('GM', 200, '01/06/1998', 56)]}
    \item Gere um relatório completo de vendas, incluindo o nome da empresa.
    \item Crie um resumo de compras que acumule o investimento total pelo símbolo da empresa.
  \end{enumerate}
\end{frame}
  \begin{frame}[t, fragile]{Funçoes}
  \begin{itemize}
      \item Blocos de códigos nomeados, concebidos para realizar uma tarefa específica
      \item Funções permitem escrever, ler, testar e corrigir os programas de modo mais fácil
  \end{itemize}  
\end{frame}
%
\begin{frame}[t, fragile]{Definindo uma função}
  \lstinputlisting[language=python]{codigos/funcoes/funcoes01.py}
\end{frame}
%
\begin{frame}[t, fragile]{Argumentos posicionais}
  \lstinputlisting[language=python]{codigos/funcoes/funcoes02.py}
\end{frame}
%
\begin{frame}[t, fragile]{Argumentos nomeados}
  \lstinputlisting[language=python]{codigos/funcoes/funcoes03.py}
\end{frame}
%
\begin{frame}[t, fragile]{Argumentos com valores default}
  \lstinputlisting[language=python]{codigos/funcoes/funcoes04.py}
\end{frame}
%
\begin{frame}[t, fragile]{Devolvendo valores simples}
  \lstinputlisting[language=python]{codigos/funcoes/funcoes05.py}
\end{frame}
%
\begin{frame}[t, fragile]{Devolvendo um dicionário}
  \lstinputlisting[language=python]{codigos/funcoes/funcoes06.py}
\end{frame}
%
\begin{frame}[t, fragile]{Passando uma lista para uma função}
  \lstinputlisting[language=python]{codigos/funcoes/funcoes07.py}
\end{frame}
%
\begin{frame}[t, fragile, allowframebreaks]{Modificando uma lista em um função}
  \lstinputlisting[language=python]{codigos/funcoes/funcoes08.py}
\end{frame}
%
\begin{frame}[t, fragile, allowframebreaks]{Evitando que a função modifique a lista}
  \lstinputlisting[language=python]{codigos/funcoes/funcoes09.py}
\end{frame}
%
\begin{frame}[t, fragile, allowframebreaks]{Armazenando as funções em módulos}
  \begin{itemize}
    \item As funções em Python podem ser armazenadas em um arquivo separado chamado de módulo
    \item O módulo pode ser importado para o programa principal utilizando o comando {\bf import}
    \item Permite ocultar detalhes de implementação e reutilizar as funções em vários programas
    \item Para criar um módulo vamos colocar a função \verb!calcula_media! no módulo cujo arquivo se chama \verb!estatistica.py!
  \lstinputlisting[language=python, caption="estatistica.py"]{codigos/funcoes/estatistica.py}
  \end{itemize}
\end{frame}
%
\begin{frame}[t, fragile]{Importando um módulo completo}
  \lstinputlisting[language=python]{codigos/funcoes/funcoes10.py}
\end{frame}
%
%
\begin{frame}[t, fragile]{Utilizando um alias para o módulo}
  \lstinputlisting[language=python]{codigos/funcoes/funcoes11.py}
\end{frame}
%
\begin{frame}[t, fragile]{Importando funções específicas}
  \lstinputlisting[language=python]{codigos/funcoes/funcoes12.py}
\end{frame}
%
\begin{frame}[t, fragile]{Utilizando um alias para uma função}
  \lstinputlisting[language=python]{codigos/funcoes/funcoes13.py}
\end{frame}
%


 
  


\end{document}

