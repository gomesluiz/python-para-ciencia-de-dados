\begin{frame}[t, fragile]{Python}
  \begin{itemize}
    \item Linguagem open-source de uso geral
    \item Orientada a objetos, funcional e procedural
    \item Interface com C/C++/Java/Fortran
    \item Diversas bibliotecas para Ciência de Dados
  \end{itemize}
\end{frame}
%
\begin{frame}[t, fragile, allowframebreaks]{Bibliotecas para Ciência de Dados}
  \begin{itemize}
    \item {\bf NumPy}: biblioteca para computação científica. Implementa arrays multidimensionais e permite a fácil execução de operações matemáticas e lógicas
    \item {\bf Matplotlib}: biblioteca para visualização e plotagem de gráficos em duas dimensões, como histogramas, barras e pizza.
    \item {\bf Pandas}: biblioteca para análise de dados. Fornece ferramentas para a manipulação de estruturas de dados, como matrizes, vetores e dataframes
    \item {\bf Scikit-learn}: biblioteca para Aprendizado de Máquina (Machine Learning). Fornece diversos algoritmos implementados, métodos de análise e processamento de dados, métricas de avaliação
    \item {\bf NLTK}: biblioteca para processamento de textos e linguagem natural. Fornece um conjunto de bibliotecas de processamento de texto para classificação, tokenização, stemming e tagging 
  \end{itemize}
\end{frame}
%

 