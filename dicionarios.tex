\begin{frame}[t, fragile]{Dicionários}
  \begin{itemize}
      \item Dicionários permitem conectar informações {\bf relacionadas} e modelar uma diversidade de objetos do mundo real
      \begin{itemize}
      \item podemos armazenar, por exemplo, o nome, ano, cor e fabricante de uma caro em uma única estrutura, ao invés de quatro estruturas separadas
     \end{itemize}
     \item Um dicionário é uma coleção de pares {\it chave-valor}. Cada chave é conectada a um valor:
     \lstinputlisting[language=python]{codigos/listas/listas33.py}
  \end{itemize}  
\end{frame}
%
\begin{frame}[t, fragile]{Acessando valores do dicionário}
  \lstinputlisting[language=python]{codigos/listas/listas34.py}
\end{frame}
%
\begin{frame}[t, fragile]{Començando um dicionário vazio}
  \lstinputlisting[language=python]{codigos/listas/listas35.py}
\end{frame}
%
\begin{frame}[t, fragile]{Um exemplo de utilização de dicionário}
  \lstinputlisting[language=python]{codigos/listas/listas36.py}
\end{frame}
%
\begin{frame}[t, fragile]{Percorrendo todos os pares chave-valor}
  \lstinputlisting[language=python]{codigos/listas/listas37.py}
\end{frame}
%
\begin{frame}[t, fragile]{Percorrendo todos as chaves}
  \lstinputlisting[language=python]{codigos/listas/listas38.py}
\end{frame}
%
\begin{frame}[t, fragile]{Percorrendo todos as chaves em ordem}
  \lstinputlisting[language=python]{codigos/listas/listas39.py}
\end{frame}
%
\begin{frame}[t, fragile]{Percorrendo todos os valores}
  \lstinputlisting[language=python]{codigos/listas/listas40.py}
\end{frame}
%
\begin{frame}[t, fragile]{Uma lista de dicionarios}
  \lstinputlisting[language=python]{codigos/listas/listas41.py}
\end{frame}
%


 