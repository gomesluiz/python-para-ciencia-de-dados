
\begin{frame}[t, fragile]{Introdução às Listas}
  \begin{itemize}
    \item Coleção de itens em uma ordem particular
    \item Itens podem ser letras, dígitos e etc.
    \begin{itemize}
      \item não precisam estar relacionados de nenhum modo em particular
    \end{itemize}
    \item Em Python, colchetes([]) indicam uma lista, e elementos individuais da
      lista são separados por vírgula
    \item A posição dos índices de uma lista começa em 0 e não em 1
  \end{itemize}
\end{frame}
%
\begin{frame}[t, fragile, allowframebreaks]{Acessando elementos da lista}
  \begin{itemize}
    \item Escreva o nome da lista seguido do índice do item entre colchetes
  
    \begin{lstlisting}[language=python]
      >>> mestres = ['yoda', 'qui-gon', 'kenobi'
          , 'luke']
      >>> print(mestres[0])
      yoda
      >>> print(mestres[0].title())
      Yoda
    \end{lstlisting}
    \item Python tem uma sintaxe especial para acessar o último elemento de uma lista
    \begin{lstlisting}[language=python]
      >>> print(mestres[-1].title())
      Luke
    \end{lstlisting}
  \end{itemize}
\end{frame}
%
\begin{frame}[t, fragile, allowframebreaks]{Acrescentando elementos no final da lista}
  \begin{itemize}
    \begin{lstlisting}[language=python]
      >>> mestres = ['yoda', 'qui-gon', 'kenobi'
          , 'luke']
      >>> mestres.append('windu')
      >>> print(mestres)
      ['yoda', 'qui-gon', 'kenobi', 'luke', 'windu']
    \end{lstlisting}
    \item O método {\bf append()} acrescenta o elemento 'windu' no final da lista 
    \pagebreak
    \item O método {\bf append()} facilita a criação de listas dinamicamente
    \begin{lstlisting}[language=python]
      casas = []
      casas.append('starks')
      casas.append('greyjoy')
      casas.append('tyrell')
      casas.append('lannister')
      print(casas)
      ['starks', 'greyjoy', 'tyrell', 'lannister']
    \end{lstlisting}
  \end{itemize}
\end{frame}
%
\begin{frame}[t, fragile]{Inserindo elementos em uma lista}
  \begin{itemize}
    \item O método {\bf insert} permite adicionar um elemento em qualquer posição da lista
    \begin{lstlisting}[language=python]
      mestres = ['yoda', 'qui-gon', 'kenobi', 'luke']
      mestres.insert(0, 'vader')
      print(mestres)
      ['vader', 'yoda', 'qui-gon', 'kenobi', 'luke']
    \end{lstlisting} 
  \end{itemize}
\end{frame}
%
\begin{frame}[t, fragile]{Removendo elementos da com {\bf del}}
  \begin{itemize}
    \item O comando {\bf del} remove item da lista 
    \begin{lstlisting}[language=python]
      mestres = ['yoda', 'qui-gon', 'kenobi', 'luke']
      del(mestres[0])
      print(mestres)
      ['qui-gon', 'kenobi', 'luke']
    \end{lstlisting} 
    \item Não é possível reutilizar o elemento removido
  \end{itemize}
\end{frame}
%
\begin{frame}[t, fragile]{Removendo elementos da lista com {\bf pop}}
  \begin{itemize}
    \item O método {\bf pop} remove o último item da lista por padrão, sendo que elemento removido pode ser reutilizado
    \lstinputlisting[language=python]{codigos/listas/listas6.py}
    \item De fato, o método {\bf pop} pode ser utilizado para remover qualquer elemento da lista, basta passar o índice do elemento
    \lstinputlisting[language=python]{codigos/listas/listas7.py}
  \end{itemize}
  
\end{frame}
%
%  \begin{frame}[t, fragile]{Removendo elementos da lista com {\bf remove}}
%    \begin{itemize}
%      \item O método {\bf remove} um elemento de acordo com seu valor.
%      \lstinputlisting[language=python]{codigos/listas/listas8.py}
%    \end{itemize}  
%  \end{frame}
%
\begin{frame}[t, fragile]{Ordenando a lista com {\bf sort}}
  \begin{itemize}
    \item O método {\bf sort} altera a forma da lista permanentemente 
    \lstinputlisting[language=python]{codigos/listas/listas9.py}
    \item O método {\bf sort} permite ordenar a lista em ordem alfabética inversa
    \lstinputlisting[language=python]{codigos/listas/listas10.py}
  \end{itemize}  
\end{frame}
%
\begin{frame}[t, fragile]{Ordenando a lista com {\bf sorted}}
  \begin{itemize}
    \item O método {\bf sorted} mantém a forma original da lista
    \lstinputlisting[language=python]{codigos/listas/listas11.py}
  \end{itemize}  
\end{frame}
%
\begin{frame}[t, fragile]{Exibindo uma lista em ordem inversa com {\bf reverse}}
  \begin{itemize}
    \item O método {\bf reverse} mantém a forma original da lista
    \lstinputlisting[language=python]{codigos/listas/listas12.py}
  \end{itemize}  
\end{frame}
%






 